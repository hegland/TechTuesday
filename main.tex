% paper on compound matrices ???
% Version 0.1
% 2016/11/01

\documentclass[12pt]{article}
\usepackage{amsmath,amssymb,amsthm,amsfonts}
\usepackage{tikz}
\usepackage{pgfplots}

% real numbers
\newcommand{\R}{\mathbb{R}}

% define lemma
\newtheorem{lemma}{Lemma}

\begin{document}

\section{Examples}

For this we use the vscode extension \texttt{LaTeX Workshop} 
and the the Copilot extension \texttt{GitHub Copilot} with 
the chat extension \texttt{GitHub Copilot Chat}.
It uses \texttt{TeXlive} and the \texttt{latexmk} command.

% a first lemma

\begin{lemma}
    All principle submatrices of a 
    symmetric positive-semidefinite matrix are
    symmetric positive-semidefinite.
\end{lemma}

\begin{proof}
    Let $A\in \R^{n,n}$ be a symmetric positive-semidefinite matrix and 
    let $B\in\R^{k,k}$ be a principle submatrix of $A$. 
    Furthermore, as $B$ is a principle submatrix of $A$, there exists
    a matrix $E\in\R^{n,k}$ such that $B = E^T A E$, 
    has only elements in $\{0,1\}$ and $E^T E=I$.
    It follows that $B$ is symmetric positive-semidefinite.
\end{proof}

GitHub Copilot: Sure, here's a clearer version of the same text:

\begin{proof}
Let $A\in \R^{n,n}$ be a symmetric positive-semidefinite matrix, 
and let $B\in\R^{k,k}$ be a principal submatrix of $A$. This 
means that $B$ is obtained by selecting $k$ rows and $k$ columns 
of $A$, and arranging them into a square matrix. Since $B$ is 
a principal submatrix of $A$, there exists a matrix 
$E\in\R^{n,k}$ such that $B = E^T A E$, where $E$ has only 
elements in $\{0,1\}$ and $E^T E=I$. This means that $B$ is 
also symmetric positive-semidefinite.
\end{proof}

% in chat: generate a csv file with 10 x and y coordinates where x 
% are random between 0 and 2 pi and y = sin(x)
% sort the results by x and save as data.csv

% read the data from data.csv and plot the data using pgfplots
% include the plot in the document as a figure

\begin{figure}
    \centering
    \begin{tikzpicture}
        \begin{axis}[
            xlabel=$x$,
            ylabel={$f(x) = \sin(x)$},
            ymin=-1, ymax=1,
            xmin=0, xmax=6.28
        ]
        \addplot table[x=x, y=y, col sep=comma] {data.csv};
        \end{axis}
    \end{tikzpicture}
    \caption{Plot of the function $f(x) = \sin(x)$}
    \label{fig:sin} 
\end{figure}


\end{document}